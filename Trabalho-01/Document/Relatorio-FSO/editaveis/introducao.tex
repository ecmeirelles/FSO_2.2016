\chapter[Introdução]{Introdução}

	Este trabalho tem como objetivo consolidar os conhecimentos adquiridos na disciplina de Fundamentos de Sistemas
	Operacionais ofertada na Universidade de Brasília, Faculdade Gama.

\section{Ambiente de Desenvolvimento}

	Para uma melhor realização do trabalho, decidiu-se por unificar, entre a dupla, as ferramentas de desenvolvimento
	utilizadas, as quais incluem sistema operacional, compilador, depurador e editor de texto. Portanto, o seguinte
	ambiente de desenvolvimento foi estabelecido:

	\begin{itemize}

		\item O sistema operacional utilizado foi o \textit{Linux} na distro Ubuntu 16.04 LTS;
		\item O código foi convertido através do compilador \textit{GCC} na versão 5.4.0;
		\item O código foi depurado, quando necessário, com o debugger \textit{GDB} na versão 7.11.1;
		\item A edição dos arquivos de código e texto foi realizada através do \textit{Sublime}.

	\end{itemize}

	Além disso, o presente documento foi contruído através do \textit{LaTeX}.

	\chapter[Implementação]{Implementação}

  A implementação das questões 1 e 2 foi contruída a partir de um \textit{Makefile} com a finalidade de automatizar a
  compilação de ambas, pois as mesmas possuem diversos ficheiros de inclusão (.h) e arquivos c (.c). Assim, para
  realizar a compilação é necessária apenas a execução do comando \textit{make} no diretório referente a questão
  (/Q1 ou /Q2). Este comando gerará o arquivo executável \textit{Q1-Triangle} e \textit{Q2-Sort} para as questões 1 e 2,
  respectivamente.

  A questão 3, entretanto, é bastante simples e não necessitou da criação do \textit{Makefile}. Desse modo, a mesma
  necessita apenas da execução do comando \textit{gcc Q3.c -o Q3-PointersAndStrings}. Este comando gerará um executável
  chamado \textit{Q3-PointersAndStrings}.

  Não foi identificada nenhuma inconsistência nas questões.

\section{Questão 1}

  Após ter gerado o executável, a realização do comando \textit{./Q1-Triangle} é necessária.

  Primeiramente, o programa necessita das coordenadas cartesianas dos vértices do triângulo desejado. As coordenadas
  serão adquiridas através de entradas digitadas pelo usuário em tempo de execução, de acordo com o que for pedido a
  cada vez. Através dessas informações a aplicação irá verificar se o triângulo é válido.

  Caso o triângulo seja válido, o programa informará:

  \begin{itemize}

    \item Valor de cada lado do triângulo;
    \item Perímetro do triângulo;
    \item Área do triângulo.

  \end{itemize}

  Caso o triângulo seja inválido, o sistemá informará a mensagem \textit{"Os valores inseridos não correspondem a um
  triângulo válido"} ao usuário.

  Para a realização da aplicação foram criados três arquivos .h (\textbf{composite\_types}, \textbf{data\_entry} e
  \textbf{geometric\_operators}) e três arquivos .c (\textbf{data\_entry}, \textbf{geometric\_operators} e
  \textbf{main}). O arquivo composite\_types.h define os pontos no plano cartesiano e o triângulo, ao passo que os
  arquivos data\_entry.h e data\_entry.c compõem as funções de entrada. geometric\_operators.h e geometric\_operators.c,
  como pode-se facilmente perceber, compreendem as funções relacionadas as operações geométricas. Finalmente, o arquivo
  main.c é onde está presente a principal função do programa: \textit{int main()}.

  \subsection{Caso de Teste: Coordenadas Válidas}

  Caso o usuário informe valores de coordenadas onde o triângulo seja válido, ou seja, o mesmo possa ser gerado, o
  programa irá executar corretamente, mostrando todas as informações. O usuário executaria os seguintes passos, já
  informando os dados:

  \begin{enumerate}
    \item Informe a 1ª coordenada x: \textbf{5};
    \item Informe a 1ª coordenada y: \textbf{5};

    \item Informe a 2ª coordenada x: \textbf{3};
    \item Informe a 2ª coordenada y: \textbf{6};

    \item Informe a 3ª coordenada x: \textbf{1};
    \item Informe a 3ª coordenada y: \textbf{4};
  \end{enumerate}


  Isto irá produzir os seguintes resultados:

  \begin{itemize}
    \item Os lados do triângulo são:
    \begin{itemize}
      \item Lado A: 2.24
      \item Lado B: 4.12
      \item Lado C: 2.83
    \end{itemize}
    \item O perímetro do triângulo é: 9.19
    \item A área de triângulo é: 3.00
  \end{itemize}

  \subsection{Caso de Teste - Coordenadas Inválidas}

  Caso o usuário informe duas coordenadas cartesianas (x, y) iguais:

  \begin{enumerate}
    \item Informe a 1ª coordenada x: \textbf{5};
    \item Informe a 1ª coordenada y: \textbf{5};

    \item Informe a 2ª coordenada x: \textbf{5};
    \item Informe a 2ª coordenada y: \textbf{5};

    \item Informe a 3ª coordenada x: \textbf{1};
    \item Informe a 3ª coordenada y: \textbf{4};
  \end{enumerate}

  Isto irá produzir o seguinte resultado:

  \begin{itemize}
    \item Valores informados não correspondem a um triângulo válido.
  \end{itemize}

\section{Questão 2}

  Após ter gerado o executável, a realização do comando \textit{./Q2-Sort \{ordem de ordenação (optativo)\} \{lista de números
  inteiros\}} é necessária.

  Primeiramente, o programa necessita de uma ordem de ordenação e/ou uma lista de números inteiros. Estas informações
  serão adquiridas através de chamada da aplicação onde tais dados serão informados como parâmetros na linha de comando.
  Há duas diferentes ordens de ordenação, são elas:

  \begin{enumerate}
    \item \textbf{-d:} Números em ordem crescente;
    \item \textbf{-r:} Números em ordem decrescente;
  \end{enumerate}

  Caso o usuário não informe o modo de ordenação, os números serão ordenados de forma crescente.

  \textbf{Exemplo.} ./Q2-Sort -d 5 2 1 3 \textbf{ou} ./Q2-Sort -r 2 1 3 4 \textbf{ou} ./Q2-Sort 5 6 2 3

  Para a realização da aplicação foram criados dois arquivos .h (\textbf{int\_list} e \textbf{sort} e três arquivos .c
  (\textbf{int\_list}, \textbf{sort} e \textbf{main}). Os arquivos int\_list.h e int\_list.c compõem a definição e
  funções relacionadas a uma lista duplamente encadeada, ao passo que os arquivos sort.h e sort.c compreendem as funções
  de ordenação crescente e decrescente. Finalmente, o arquivo main.c é onde está presente a principal função do
  programa: \textit{int main()}.

  \textbf{OBSERVAÇÃO.} Foi escolhida uma lista duplamente encadeada para a implementação da solução em virtude de sua
  eficiência em diferentes modos de ordenação, já que possui ponteiros para nós anteriores e posteriores, assim como sua
  capacidade de alocação dinâmica na memória, o que dispensa o uso desnecessário de seus espaços.

  \subsection{Caso de Teste: Modo de Ordenação -d}

  Caso o usuário execute a aplicação passando a ordem de ordenação -d, ou seja, crescente, juntamente com os números
  desejados, a chamada da aplicação deverá ser assim:

  \begin{enumerate}
    \item \$ ./Q2-Sort -d 8 3 5 2 6 1 4 7
  \end{enumerate}


  Isto irá produzir o seguinte resultado:

  \begin{itemize}
    \item Lista = \{ 1 2 3 4 5 6 7 8 \}
  \end{itemize}

  \subsection{Caso de Teste: Modo de Ordenação -r}

  Caso o usuário execute a aplicação passando a ordem de ordenação -r, ou seja, decrescente, juntamente com os números
  desejados, a chamada da aplicação deverá ser assim:

  \begin{enumerate}
    \item \$ ./Q2-Sort -r 8 3 5 2 6 1 4 7
  \end{enumerate}


  Isto irá produzir o seguinte resultado:

  \begin{itemize}
    \item Lista = \{ 8 7 6 5 4 3 2 1 \}
  \end{itemize}

  \subsection{Caso de Teste: Sem Modo de Ordenação}

  Caso o usuário execute a aplicação sem passar a ordem de ordenação, o programa entenderá como um desejo por ordem
  crescente dos números desejados, a chamada da aplicação deverá ser assim:

  \begin{enumerate}
    \item \$ ./Q2-Sort 4 2 5 3 1
  \end{enumerate}


  Isto irá produzir o seguinte resultado:

  \begin{itemize}
    \item Lista = \{ 1 2 3 4 5 \}
  \end{itemize}

    \subsection{Caso de Teste: Chamada sem Parâmetros}

  Caso o usuário tente executar a aplicação sem passar nada como parâmetro na linha de comando, o programa não executará
  e mostrará uma mensagem de erro. Se a chamada da aplicação for assim:

  \begin{enumerate}
    \item \$ ./Q2-Sort
  \end{enumerate}


  Isto irá produzir o seguinte resultado:

  \begin{itemize}
    \item \{ERROR\} Voce precisa adicionar a ordem de ordenação e/ou os numeros desejados.
  \end{itemize}

\section{Questão 3}

