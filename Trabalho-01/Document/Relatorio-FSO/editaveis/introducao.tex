\chapter[Introdução]{Introdução}

	Este trabalho tem como objetivo consolidar os conhecimentos adquiridos na disciplina de Fundamentos de Sistemas Operacionais.

\section{Informações gerais}

	Essa seção informa detalhes do ambiente de desenvolvimento e as ferramentas utilizadas.

\subsection{Ambiente de desenvolvimento}

	O sistema operacional utilizado foi o \textit{Linux} na distro Ubuntu 14.04. O \textit{Sublime} foi usado para a edição dos arquivos e o compilador foi o \textit{gcc} na versão  4.8.4.

\section{Questões}

	Para realizar a compilação das questões 1 e 2 execute o comando \textit{make} no diretório referente a questão. Esse comando irá gerar o arquivo executável \textit{triangle} para a questão 1 e \textit{Q2-Sort} para a questão 2.

	Para gerar o executável da questão 3 execute o seguinte comando: gcc Q3.c -o Q3
	Com isso será gerado um executável chamado Q3.

	Não foi identificado nenhuma inconsistência nas questões.

\subsection{Questão 1}

	Após ter gerado o executável, rode o comando: ./triangle

	O programa ira lhe pedir informações referentes as coordenadas do triangulo e depois irá validar se o triangulo é válido. Caso o triângulo seja um triângulo válido o programa informará: Os valor dos lados do triângulo, o perímetro e a área do triangulo. Caso o triângulo seja inválido, o sistemá informará ao usuário que os valores inseridos não correspondem a um triângulo válido.

	\textbf{Caso de Teste válido:}


		1 Ponto x: 5; 1 Ponto y: 5; 2 Ponto x: 3; 2 Ponto y: 6; 3 Ponto x: 1; 3 Ponto y: 4.

		Irá produzir os seguintes resultados:

	 Os lados do triângulo são:

		Lado A: 2.24, Lado B: 4.12, Lado C: 2.83

	 O perímetro do triângulo é: 9.19

	 A área de triângulo é: 3.00

	\textbf{Caso de Teste inválido:}

		Um caso em que dois pontos informados possuem as mesmas coordenadas:

		1 Ponto x: 5; 1 Ponto y: 5; 2 Ponto x: 5; 2 Ponto y: 5; 3 Ponto x: 1; 3  Ponto y: 4;

		O resultado será:

		Valores informados não correspondem a um triângulo válido.

\subsection{Questão 2}
\subsection{Questão 3}





